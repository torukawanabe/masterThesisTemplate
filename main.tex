% 独自のコマンド
% ■ 謝辞
%  \begin{acknowledgment} 〜 \end{acknowledgment}

\documentclass[uplatex,report]{jsbook}
\newcommand{\acknowledgmentname}{謝辞}

\usepackage{thesis}
\usepackage[dvipdfmx]{graphicx}
\usepackage{multirow}
\usepackage{url}
\usepackage{multibbl}
\usepackage{float}
\usepackage{ascmac}
\usepackage{array,booktabs}
\usepackage[labelsep=colon,hang]{caption}

\newbibliography{biblio}
\newbibliography{publ}

\setcounter{tocdepth}{2}

% 日本語情報(必要なら)
\jclass  {修士論文}                             % 論文種別
\jtitle    {実世界に爆発を重畳させて\\対象物を破壊するディスプレイ
}    % タイトル。改行する場合は\\を入れる
\juniv    {早稲田大学理工学術院}                  % 大学名
\jfaculty  {表現工学科}               % 学部、学科
\jauthor  {東京 花子}                       % 著者
\jsyear  {20XX}                                 % 西暦○年度
\jkeyword  {ディスプレイ、爆発}     % 論文のキーワード
\jdate{20XX年Y月Z日} % 提出日
\jadvisor{早稲田 表現 准教授} % 指導教員
\jnumber{5114E004-3} % 学籍番号

%hyperrefパッケージは影響が大きいのでプリアンブルの最後尾で呼び出し
\usepackage[dvipdfmx]{hyperref}
\usepackage{pxjahyper}

\begin{document}

\jmaketitle

\begin{jabstract}

有史以降,爆発は様々な用途に用いられてきたが,インタフェースの分野での利用は存在しない.本研究では爆発を対象物に重畳することでディスプレイとすることとした.本論文では、爆発ディスプレイの設計、実装、アプリケーションについて報告する。

\end{jabstract}  % アブストラクト。要独自コマンド、include先参照のこと

\tableofcontents  % 目次
\newpage
\listoffigures    % 図目次
\newpage
\listoftables    % 表目次
\newpage

\pagenumbering{arabic}

\chapter{序論}
\label{chap:introduction}
\section{背景と目的}

有史より人間は爆発に興味を持ち、任意に爆発させる仕組みを開発してきた。ノーベルによるダイナマイトはその代表的な例\cite{biblio}{nobel2003}である.それらの研究は爆発の制御が主であり,ディスプレイに応用した例はない.そこで本研究では爆発を対象物に重畳することでディスプレイとすることとした。

\section{本論文の構成}

第\ref{chap:introduction}章では研究の背景と目的を述べた.

第\ref{chap:relatedworks}章では,本研究の関連研究をまとめ,本研究の位置付けを述べる.

第\ref{chap:system}章では,爆発の制御によるディスプレイ 爆発ディスプレイ に関してまとめる.爆発の発生制御の仕組みと,設計,実装,について述べる.

第\ref{chap:application}章では,爆発ディスプレイのアプリケーションについて述べる.

第\ref{chap:conclusion}章では,全体のまとめと,将来展望を述べる.
  % 本文1
\chapter{関連研究}
\label{chap:relatedworks}

\section{爆発}

有史以来,人類は様々な爆発を開発してきた.本節では以下,破壊のための爆発,人的被害を与えるための爆発,演出のための爆発,その他の爆発の4種類に分けて紹介する.

\subsection{破壊のための爆発}

最も基本的な爆発である.対象を破壊させるための爆発である.

\subsection{人的被害を与えるための爆発}

対象を破壊させるのではなく人的被害を最大化させるための爆発である.主に戦争に利用される.

\subsection{演出のための爆発}

今日の映像作品において,爆発は一般的に利用されている.それは演出のための爆発である.

\subsection{その他の爆発}

宇宙空間での爆発は重力がないゆえに,地球上での爆発と異なる性質を持つ.

\section{本研究の位置づけ}

既存の爆発と本研究との違いについて述べる.本章で紹介した爆発と本研究で提案するシステムの違いを,表にまとめた(表\ref{explosionDiff}).

\begin{table}[H]
	\caption{既存の爆発と爆発ディスプレイの違い}
	\label{explosionDiff}
	\begin{center}
	\setlength{\tabcolsep}{6pt}
	\footnotesize
	\begin{tabular}{ll} \toprule
		\multicolumn{1}{c}{爆発手法} & \multicolumn{1}{c}{特徴} \\ \midrule
		既存の爆発 & ただ爆発する.用途に合わせた爆発がある \\
		爆発ディスプレイ & 爆発をディスプレイとして利用るため超繊細な制御 \\ \bottomrule
	\end{tabular}
	\end{center}
\end{table}

本研究の位置づけをまとめる.これまで爆発はディスプレイを目的として開発されなかった.本研究では,爆発をディスプレイとして利用する.  % 本文2
\chapter{爆発ディスプレイ}
\label{chap:system}

\section{概要}

本研究で提案する爆発ディスプレイは爆発を対象物に重畳することで対象物を破壊して見た目を変化させるディスプレイである(図\ref{fig:systemImage}\footnote{出典:文献\cite{biblio}{bakuhatsu2016}より}).

\easyfig{img/bakuhatsu.png}{爆発ディスプレイ イメージ図}{fig:systemImage}{H}


\section{設計}

対象物にダイナマイトを巻きつけ、導火線を引き、手元のスイッチで爆発させる仕組みである.

\section{実装}

今回はスマートフォン(Apple製 iPhone6)に対してダイナマイトを巻きつけた.  % 本文3
\chapter{アプリケーション}
\label{chap:application}

爆発ディスプレイのアプリケーションとして,秘密秘匿装置を提案する.秘密に属する通信文の入った文書やデバイス等に遠隔操作装置をつけた本システムを装着し,通信を相手に伝えたと同時に爆発する.これにより,秘密の保持と相手への威圧が同時に行うことが出来る.
  % 本文4
\chapter{まとめと今後の展望}
\label{chap:conclusion}

\section{本研究のまとめ}

本論文では,爆発を対象物に重畳する爆発ディスプレイを提案・実装した.また,アプリケーションとして秘密秘匿装置を提案した.

\section{今後の展望}

今後の展望としては,実装したと書いたが実装していないため,実装が待たれるが、これはただコピペして使うことがないようにとの配慮のために絶対に実装されないシステムを書いたまでで,爆発を素子としたディスプレイなど作らないでくれ.  % 本文5

\begin{acknowledgment}

本テンプレート改変元を作っていた@ymrl及びそのまた改変元である@kurkoboの両名に感謝する.また,それ以外に参考にした全ての文献の作成者に感謝する.

\end{acknowledgment}
  % 謝辞。要独自コマンド、include先参照のこと

\nocite{biblio}{*}
\nocite{publ}{*}

\bibliographystyle{biblio}{jplain}
\bibliography{biblio}{bibDB}{参考文献}

\bibliographystyle{publ}{jplain}
\bibliography{publ}{pubDB}{発表文献}

\end{document}