\chapter{関連研究}
\label{chap:relatedworks}

\section{爆発}

有史以来,人類は様々な爆発を開発してきた.本節では以下,破壊のための爆発,人的被害を与えるための爆発,演出のための爆発,その他の爆発の4種類に分けて紹介する.

\subsection{破壊のための爆発}

最も基本的な爆発である.対象を破壊させるための爆発である.

\subsection{人的被害を与えるための爆発}

対象を破壊させるのではなく人的被害を最大化させるための爆発である.主に戦争に利用される.

\subsection{演出のための爆発}

今日の映像作品において,爆発は一般的に利用されている.それは演出のための爆発である.

\subsection{その他の爆発}

宇宙空間での爆発は重力がないゆえに,地球上での爆発と異なる性質を持つ.

\section{本研究の位置づけ}

既存の爆発と本研究との違いについて述べる.本章で紹介した爆発と本研究で提案するシステムの違いを,表にまとめた(表\ref{explosionDiff}).

\begin{table}[H]
	\caption{既存の爆発と爆発ディスプレイの違い}
	\label{explosionDiff}
	\begin{center}
	\setlength{\tabcolsep}{6pt}
	\footnotesize
	\begin{tabular}{ll} \toprule
		\multicolumn{1}{c}{爆発手法} & \multicolumn{1}{c}{特徴} \\ \midrule
		既存の爆発 & ただ爆発する.用途に合わせた爆発がある \\
		爆発ディスプレイ & 爆発をディスプレイとして利用るため超繊細な制御 \\ \bottomrule
	\end{tabular}
	\end{center}
\end{table}

本研究の位置づけをまとめる.これまで爆発はディスプレイを目的として開発されなかった.本研究では,爆発をディスプレイとして利用する.